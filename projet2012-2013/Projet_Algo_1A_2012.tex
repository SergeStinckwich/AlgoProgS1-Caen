\documentclass[10pt]{article}
\usepackage[utf8]{inputenc}
\usepackage{times}
\usepackage[french]{babel}
\usepackage{latexsym}
\usepackage[T5,T1]{fontenc}

% Packages
%----------
\ExecuteOptions{textures}
%\usepackage{epic,eepic}
\usepackage{epsfig}
%\usepackage{fullpage}

\setlength{\textwidth}{150mm}     %
\setlength{\oddsidemargin}{10mm}  % Please
\setlength{\topmargin}{0mm}       %
\setlength{\textheight}{200mm}    % do not
\setlength{\headheight}{0mm}     %
\setlength{\headsep}{0mm}        % modify
\setlength{\topskip}{0in}         %
\def\thepage{}
% Macros
%----------
%\setlength{\textwidth}{}
\setlength{\textheight}{240mm}
\newcommand{\COMMENT}[1]{}

%% Le titre du papier
\title{Projet 1A : \textsc{Schildkrötenrennen \\ La course des tortues }}
\author{IUT CAEN -- Département Informatique}

\begin{document}
\maketitle
\bibliographystyle{plain}
\pagestyle{empty}

\section{Description du projet}

Il s'agit d'un jeu de plateau dont les règles sont suffisamment
simples pour être joué à partir de 3 ans : La Course des Tortues.

Le plateau de jeu est compos\'e de $10$ cases dont un d\'epart et une
arriv\'ee (au milieu d'un champ de laitues, par exemple). Il y a $5$
tortues de couleurs diff\'erentes (vert, jaune, bleu, rouge, violet).
Chaque joueur se voit attribuer une tortue, mais il est le seul \`a
savoir quelle est sa tortue. Il poss\`ede \'egalement $5$ cartes en main.
Toutes les tortues, m\^eme celles qui ne sont affect\'ees \`a aucun
joueur sont mises en jeu.

Les $72$ cartes ont une couleur et une valeur. Les valeurs sont $+2$, $+1$
et $-1$. Les couleurs sont celles des tortues et une couleur sp\'eciale
correspondant \`a la derni\`ere tortue. Leur r\'epartition par couleur (y
compris la couleur sp\'eciale) est la suivante :
\begin{itemize}
\item $5$ cartes $+2$
\item  $5$ cartes $+1$
\item $2$ cartes $-1$
\end{itemize}

Chaque joueur joue une carte \`a son tour et avance (ou recule) la
tortue correspondant \`a la couleur de la carte (ou la ou les derni\'ere
des tortues pour la couleur sp\'eciale) du nombre de case sp\'ecifi\'e
par la valeur de la carte (1 ou deux cases). Une nouvelle carte est
pioch\'ee \`a la fin de chaque tour.

L'astuce est que les tortues partent superpos\'ees et qu'une tortue qui
avance emporte toutes celles qui sont sur son dos. Lorsqu'une tortue
arrive sur une case occup\'ee, elle est plac\'ee sur le dos de la tortue
qui \'etait d\'ej\`a sur la case.

Le jeu s'arr\^ete lorsqu'une tortue arrive ou d\'epasse la case
d'arriv\'ee. Le joueur poss\'edant la tortue la plus proche de
l'arriv\'ee a gagn\'e. En cas d'\'egalit\'e, c'est la tortue la plus en
dessous qui gagne.


\section{Fonctionnalit\'es \`a impl\'ementer}

Votre programme devra permettre de visualiser une s\'equence de jeu
en affichant \`a chaque tour : le joueur, la carte jou\'ee et les
positions des tortues apr\`es chaque tour (Joueur1, Carte rouge +2,
Tortue Jaune : case 2, Tortue Rouge : Case 2 sur Tortue Jaune, Tortue
Verte : Case D?part, etc). 

On pourra faire une fonction qui \`a chaque tour choisira une carte
al\'eatoirement dans le jeu du joueur et d\'eplacera les tortues en
cons\'equence en montrant textuellement la position des tortues apr\`es
leur d\'eplacement. Ne pas oublier que l'on pioche \`a chaque tour pour
avoir une main compl\`ete au prochain tour. Les cartes jou\'ees sont
r\'einject\'ees sous la pioche.


Pour ceux qui veulent aller plus loin, vous pouvez r\'ealiser au choix
: des graphiques avec animation, une version en 3D, une version pour
jouer \`a plusieurs en r\'eseau, une version pour jouer \`a plusieurs sur
un site web.


\section{Notes de mise en {\oe}uvre}

\section{Travail \`a rendre}
Le projet est \`a r\'ealiser en bin\^omes (ou, avec l'accord de votre 
charg\'e de TP, en mon\^ome). Le travail \`a rendre est un projet sous forme 
d'une archive zip {\bf \`a envoyer par mail \`a votre charg\'e de TP}. 

\medskip
\noindent
Le nom de l'archive doit avoir la forme suivante~: \texttt{Nom1Nom2.grTP.zip} ou 
\texttt{Nom1.grTP.zip} o\`u \texttt{Nom1} et \texttt{Nom2} sont les noms de 
famille des membres des polyn\^omes et \texttt{grTP} est le nom du groupe de TP 
auquel ils appartiennent (1.1, 1.2, etc.). 

\medskip
\noindent
{\bf Note}: Lors des s\'eances de TP, les charg\'es de TP suivront l'avancement de votre projet et pourront vous aider sur certains points difficiles.

\subsection*{Ce qu'il faut rendre~:}
\begin{itemize}
\item Le code source complet de votre application en C largement comment\'e.

\item Un ex\'ecutable test\'e \textbf{sous Linux} et op\'erationnel avec sa documentation d'installation. 

\underline{\bf ATTENTION} : 
si votre programme fait appel \`a d'autres librairies externes (comme par exemple 
les librairies du SDL), il est imp\'eratif de les inclures pour les besoins de test. 

\item Un court rapport d'une longueur comprise entre 3 et 10 pages pr\'esentant :
  \begin{itemize}
    \item les fonctionnalit\'es impl\'ement\'ees (tr\`es bri\`evement).
    \item organisation du programme : d\'ecoupage en fonctions, r\^ole de 
      ces fonctions, explications du programme. 
    \item l'organisation et la r\'epartition des t\^aches au sein du bin\^ome 
      durant la dur\'ee du projet (bri\`evement).
    \item bilan qualitatif du travail, difficult\'es rencontr\'ees, points qui 
      vous ont paru int\'eressants. 
    \item un mode d'emploi avec quelques illustrations (p. ex. capture d'\'ecrans, 
      sc\'enario d'ex\'ecution...), destin\'ees \`a montrer l'op\'erationnalit\'e 
      de votre application.
    \item Une conclusion sur l'apport (ou non) du projet en termes technique,
      scientifique, humain. 
  \end{itemize}

\textbf{Le code source ne doit pas faire partie du rapport d'une dizaine de pages 
(sinon en annexe)}.
\end{itemize}

\section{\'Evaluation du projet et calendrier}

Le projet est \`a rendre le \underline{\bf vendredi 11 janvier 2013} 
(tout retard conduira \`a des p\'enalit\'es). L'\'evaluation sera r\'ealis\'ee en focntion 
des crit\`eres ci-dessous~:
\begin{itemize}

\item  \textit{qualit\'e technique du code}~: d\'ecoupage en fonctions, modularit\'e et r\'eutilisabit\'e, instructions, algorithmes, efficacit\'e, gestion des erreurs lors de d'une saisie, ou lorsque les donn\'ees fournies en param\`etres des fonctions sont incorrectes.

\item \textit{lisibilit\'e du code}~: pr\'esentation du programme (indentation), usage de variables ayant des noms explicites, commentaires pour pr\'eciser les points difficiles dans les algorithmes, param\`etres des fonctions $\ldots$ 

\item \textit{documentation fournie}~: organisation du programme et son mode d'emploi, bilan.

\item \textit{pr\'esentation orale}~: d\'emonstration du programme et questions sur le travail r\'ealis\'e.


\end{itemize}
\end{document}
