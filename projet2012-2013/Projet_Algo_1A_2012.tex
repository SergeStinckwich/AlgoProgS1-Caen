\documentclass[10pt]{article}
\usepackage[utf8]{inputenc}
\usepackage{times}
\usepackage[french]{babel}
\usepackage{latexsym}
\usepackage[T5,T1]{fontenc}

% Packages
%----------
\ExecuteOptions{textures}
%\usepackage{epic,eepic}
\usepackage{epsfig}
%\usepackage{fullpage}

\setlength{\textwidth}{150mm}     %
\setlength{\oddsidemargin}{10mm}  % Please
\setlength{\topmargin}{0mm}       %
\setlength{\textheight}{200mm}    % do not
\setlength{\headheight}{0mm}     %
\setlength{\headsep}{0mm}        % modify
\setlength{\topskip}{0in}         %
\def\thepage{}
% Macros
%----------
%\setlength{\textwidth}{}
\setlength{\textheight}{240mm}
\newcommand{\COMMENT}[1]{}

%% Le titre du papier
\title{Projet 1A : \textsc{Algorithmique-Programmation \\ La course des tortues }}
\author{IUT CAEN -- Département Informatique}

\begin{document}
\maketitle
\bibliographystyle{plain}
\pagestyle{empty}

\section{Description du projet}

Il s'agit d'un jeu de plateau dont les règles sont suffisamment
simples pour être joué à partir de 3 ans : La Course des Tortues.

Le plateau de jeu est composé de $10$ cases dont un départ et une
arrivée (au milieu d'un champ de laitues, par exemple). Il y a $5$
tortues de couleurs différentes (vert, jaune, bleu, rouge, violet).
Chaque joueur se voit attribuer une tortue, mais il est le seul à
savoir quelle est sa tortue. Il possède également $5$ cartes en main.
Toutes les tortues, même celles qui ne sont affectées à aucun
joueur sont mises en jeu.

Les $72$ cartes ont une couleur et une valeur. Les valeurs sont $+2$, $+1$
et $-1$. Les couleurs sont celles des tortues et une couleur spéciale
correspondant à la dernière tortue. Leur répartition par couleur (y
compris la couleur spéciale) est la suivante :
\begin{itemize}
\item $5$ cartes $+2$
\item  $5$ cartes $+1$
\item $2$ cartes $-1$
\end{itemize}
Chaque joueur joue une carte à son tour et avance (ou recule) la
tortue correspondant à la couleur de la carte (ou la ou les dernières
des tortues pour la couleur spéciale) du nombre de case spécifié
par la valeur de la carte (1 ou deux cases). Une nouvelle carte est
piochée à la fin de chaque tour.

L'astuce est que les tortues partent superposées et qu'une tortue qui
avance emporte toutes celles qui sont sur son dos. Lorsqu'une tortue
arrive sur une case occupée, elle est placée sur le dos de la tortue
qui était déjà sur la case.

Le jeu s'arrête lorsqu'une tortue arrive ou dépasse la case
d'arrivée. Le joueur possédant la tortue la plus proche de
l'arrivée a gagné. En cas d'égalité, c'est la tortue la plus en
dessous qui gagne.

\section{Fonctionnalités à implémenter}
Votre programme devra permettre de visualiser une séquence de jeu
en affichant à chaque tour : le joueur, la carte jouée et les
positions des tortues après chaque tour (Joueur1, Carte rouge +2,
Tortue Jaune : case 2, Tortue Rouge : Case 2 sur Tortue Jaune, Tortue
Verte : Case Départ, etc).

On pourra faire une fonction qui à chaque tour choisira une carte
aléatoirement dans le jeu du joueur et déplacera les tortues en
conséquence en montrant textuellement la position des tortues après
leur déplacement. Ne pas oublier que l'on pioche à chaque tour pour
avoir une main complète au prochain tour. Les cartes jouées sont
réinjectées sous la pioche.

Pour ceux qui veulent aller plus loin, vous pouvez réaliser au choix
: des graphiques avec animation, une version en 3D, une version pour
jouer à plusieurs en réseau, une version pour jouer à plusieurs sur Internet.

\section{Notes de mise en {\oe}uvre}

Dans un premier temps, définissez les principales structures de données dont vous allez avoir besoin pour représenter les éléments du jeu: carte, tortue, joueur, pioche, plateau de jeu et gestionnaire de jeu (contenant les informations globales concernant l'état du jeu courant). Pour chaque structure de données, définir les données correspondantes: par exemple une carte est définie par une valeur et une couleur et les fonctions associées : par exemple la fonction jouerUneCarte(j) pour le joueur j ou piocherCarte(p) pour piocher une carte dans la pioche p.

Dans un deuxième temps, définir la boucle de contrôle globale du jeu qui assurent les différentes phases du jeu.

Pour les besoins du jeu, il pourra être utile de définir un joueur aléatoire qui joue aléatoirement en fonction des cartes dont il dispose.

Il est conseillé d'avoir en permanence un programme qui compile et qui définit une version partielle du jeu.

Si vous souhaitez développer une version graphique de votre programme, il est préférable de définir une version non graphique afin de bien séparer les fonctions associées au jeu et l'affichage graphique du jeu.

\section{Travail à rendre}
Le projet est à réaliser en binômes (ou, avec l'accord de votre 
chargé de TP, en monôme). Le travail à rendre est un projet sous forme 
d'une archive zip {\bf à envoyer par mail à votre enseignant chargé de TP}. 

\medskip
\noindent
Le nom de l'archive doit avoir la forme suivante~: \texttt{Nom1Nom2.grTP.zip} ou 
\texttt{Nom1.grTP.zip} où \texttt{Nom1} et \texttt{Nom2} sont les noms de 
famille des membres des polynômes et \texttt{grTP} est le nom du groupe de TP 
auquel ils appartiennent (1.1, 1.2, etc.).

\medskip
\noindent
{\bf Note}: Lors des séances de TP, les enseignants chargés de TP suivront l'avancement de votre projet et pourront vous aider sur certains points difficiles.

\subsection*{Ce qu'il faut rendre~:}
\begin{itemize}
\item Le code source complet de votre application en C largement commenté.

\item Un exécutable test\'e \textbf{sous Linux} et opérationnel avec sa documentation d'installation et d'utilisation. 

\underline{\bf ATTENTION} :
si votre programme fait appel à d'autres librairies externes (comme par exemple 
la librairie SDL), il est impératif de les inclure pour les besoins de test. 

\item Un court rapport d'une longueur comprise entre 3 et 10 pages présentant :
  \begin{itemize}
    \item les fonctionnalités implémentées (très brièvement).
    \item organisation du programme : découpage en fonctions, rôle de 
      ces fonctions, explications du programme. 
    \item l'organisation et la répartition des tâches au sein du binôme 
      durant la durée du projet (brièvement).
    \item bilan qualitatif du travail, difficultés rencontrées, points qui 
      vous ont paru intéressants. 
    \item un mode d'emploi avec quelques illustrations (p. ex. capture d'écrans, 
      scénario d'exécution...), destinées à montrer l'opérationnalité 
      de votre application.
    \item Une conclusion sur l'apport (ou non) du projet en termes technique,
      scientifique, humain. 
  \end{itemize}

\textbf{Le code source ne doit pas faire partie du rapport d'une dizaine de pages 
(sinon en annexe)}.
\end{itemize}

\section{Évaluation du projet et calendrier}

Le projet est à rendre le \underline{\bf vendredi 11 janvier 2013} 
(tout retard conduira à des pénalités). L'évaluation sera réalisée en fonction 
des critères ci-dessous~:
\begin{itemize}

\item  \textit{qualité technique du code}~: découpage en fonctions, modularité et réutilisabité, instructions, algorithmes, efficacité, gestion des erreurs lors de saisies, ou lorsque les données fournies en paramètres des fonctions sont incorrectes.

\item \textit{lisibilité du code}~: présentation du programme (indentation), usage de variables et de fonctions ayant des noms explicites, commentaires pour préciser les points difficiles dans les algorithmes, paramètres des fonctions $\ldots$ 

\item \textit{documentation fournie}~: organisation du programme et son mode d'emploi, bilan.

\item \textit{présentation orale}~: démonstration du programme et questions sur le travail r\'ealis\'e.


\end{itemize}
\end{document}
